\documentclass[12pt]{report}

\usepackage[utf8]{inputenc}
\usepackage[T1]{fontenc}
\usepackage[francais]{babel}
\usepackage{amsmath}
\usepackage{amsfonts}
\usepackage{amssymb}
\usepackage{amsopn}
\usepackage{mathrsfs}
\usepackage[hidelinks]{hyperref}

\usepackage[bottom=2cm,left=2cm,right=2cm,top=2cm]{geometry}

\title{Rapport sur Super OS}

\author{Thibaut Pérami, Théophile Wallez}


\begin{document}

\maketitle

\tableofcontents

\addcontentsline{toc}{chapter}{\protect\numberline{}Introduction}
\chapter*{Introduction}

Super OS est système d'exploitation expérimental(i.e où le BSOD (Blue Screen Of
Death) est une fonctionnalité utilisée fréquemment). Il tourne sur
l'architecture x86\_64 ou AMD64. Il boot selon la spécification multiboot,
Il est capable de gérer un unique disque dur avec une table de partition MBR
(partition étendue non suportée) et le système de fichier ext2. Il gère le
clavier et la souris par l'interface PS/2 et l'écran par l'interface VBE.
L'architecture n'est pas très innovante : un énorme noyau monolithique comprenant
tous les systèmes, y compris le gestionnaire de fenêtre. Nous n'avons
malheureusement pas eu le temps de développer d'interface réseau ni
d'architecture multi-coeur.

Le présent rapport ne décrit le système que d'un point de vue global, pour toute
forme de détails consulter la documentation (\verb$make doc$ à la racine puis
\verb$index.html$). Le premier chapitre décrit notre interface avec le matériel,
le second notre interface avec les programmes utilisateur, le troisième le
logiciel développés pour Super OS et le dernier les problèmes rencontrés pendant
sa conception.

Le langage utilisé est le \verb$C++$ à part quelques fichiers en assembleur, et
5 lignes de python (générant de l'assembleur :-) ). Le projet atteint les $13 000$
lignes de code, (\verb$make count$ pour les statistiques exactes).

Plan de l'archive :
\begin{itemize}
\item \verb$src$ : Le noyau lui-même en 64 bits
\item \verb$src32$ : Dernier \textit{stage} du boot, loader fait maison pour
  booter notre noyau.
\item \verb$libc$ : Notre bibliothèque \og standard \fg{} du C (on est assez
  loin du standard en fait)
\item \verb$libc++$ : Notre bibliothèque \og standard \fg{} du C++
\item \verb$doc$ : La documentation (\verb$make doc$ pour générer).
\item \verb$unitTests$ : Les tests unitaire (\verb$make unittest$ pour lancer)
\item \verb$user$ : Tout les programmes user-mode
  \begin{itemize}
  \item \verb$init$ : Processus init du système, sert de racine à l'arbre des
    processus.
  \item \verb$start$ : Fichier de démarrage du runtime C/C++ de l'user-mode.
  \item \verb$ttsh$ : Notre shell.
  \end{itemize}
\end{itemize}


\chapter{Hardware}

\section{Boot : convention multiboot}

Nous souhaitions respecter la convention multiboot \cite{multiboot}, pour
pouvoir utiliser un bootloader existant comme GRUB. Cette specification ne
traite que de système d'exploitation x86 i.e 32 bits, nous devions donc trouver
un moyen de faire booter notre 64 bits avec.

Il est vite apparut très dur de maintenir du code 32 bits et du code 64 bits
dans le même exécutable, nous avons donc décider de séparer. Nous avons donc un
premier exécutable, le loader en 32 bits qui reçoit le binaire du noyau comme
module multiboot. Le loader va alors parser le noyau pour en déduire comment il
doit le mapper dans la mémoire virtuelle puis va le lancer.

Le loader parse également les information envoyer par le boot loader multiboot,
comme la taille de la RAM, ou les information graphiques, pour vérifier qu'elle
sont valides et les passer au noyau.

Le détail de la séquence de boot sont dans la page \verb$Booting Sequence$ de la
documentation.

\section{CPU : modes et droits}

Le processeur \verb$x86_64$ a un certain nombre de mode d'exécution décrit
section 1.3 de \cite{specAMD} :
\begin{itemize}
\item Real Mode : mode au démarrage, code 16-bit, 1 Mo d'espace addressable, pas
  de mémoire virtuelle.
\item Protected Mode : mode de la specification multiboot (GRUB nous lance dans
  ce mode) : code 32 bits, 4 Go d'espace addressable virtuelle et physique
\item Long Mode, Compatibilité 32 bits : mode dans lequel la majorité du code du
  loader est exécuté : code 32 bits, 4 Go d'espace addressable virtuel
\item Long Mode : mode du noyau : code 64 bits, 256 To d'espace addressable
  virtuel, 64 Po d'espace addressable physique. 
\end{itemize}

le bootloader se charge d'initialiser le processeur en mode protégé, le loader
va alors, après un petite phase d'initialisation, passer en mode Compatibilité
32 bit, où il va s'exécuter. Il vérifie que le Long Mode est bien actif sur le
processeur courant puis juste avant de lancer le noyau, il passe en
Long Mode. 

En dehors des modes processeur, le programme courant a un niveau de droit
appelé CPL. Il va de 0 à 3, 0 étant l'état superuser où l'on a tous les droits
et 3 étant l'user-mode pour les applications. Super OS n'utilise que les 2
niveau de droits extrêmes (pas 1 et 2).

\section{RAM : pagination et droits}

\paragraph{Comparaison avec le 32 bits :} L'architecture \verb$x86$ supportait 2
mode de protection de la mémoire, la segmentation (découper la mémoire en
segments) et la pagination (avoir un espace mémoire virtuel mapper à volonté
dans la mémoire physique).

\paragraph{En Long Mode}, la segmentation est désactivée, les segments et registres de
segmentation ne servent plus que pour gérer les droits.
En revanche, la pagination est obligatoire, il est impossible de faire tourner
un code long mode sans pagination, elle doit donc être mise en place dans un
autre mode : c'est ce que fait le loader.

La pagination permet d'avoir un adressage virtual de 256 To (En réalité dû a une
simplification d'implémentation, Super OS n'est capable d'adresser qu'un seul
To) et d'envoyer a volonté n'importe quelle adresse virtuelle sur la zone
physique de son choix.
Les détails du fonctionnement sont dans la page \verb$Paging$ de la
documentation.

\paragraph{Découpage de l'espace virtuel :} Nous avons appliqué un découpage
très simple : les adresses positives de 0 à 512 Go sont disponible pour chaque
programme. Les adresses de -512 Go à 0 sont celle du noyau.

Les détails sont dans la page \verb$Virtual mappings$ de la documentation.

\section{HDD : Interface ATA et ext2}

\paragraph{Disque Dur :} Nous utilisons l'interface ATA PIO cf \cite{ATAosdev}.
Nous utilisons Seulement 2 commandes : Read LBA et write LBA sur 48 bits.
La plupart des BIOS ont une option pour émuler un driver ATA PIO sur un
disque dur SATA, donc malgré la lenteur, cela ne pose pas de problème.

\paragraph{Système de fichier :} Nous avons commencé par faire un driver FAT en lecture seule, en se disant que c'est un système de fichier simple, et donc que ça serait relativement agréable à coder.
Grave erreur.
Nous avons donc décidé de nous tourner vers un système de fichier qui nous agresse moins intellectuellement : l'ext2.

Nous gérons donc la lecture et l'écriture de fichiers et dossiers. Nous ne gérons pas les liens symboliques.
Nous ne gérons pas non plus les différentes dates associées à chaque fichiers, puisque nous ne gérons pas le décodage de l'horloge CMOS.

Nous avons fait des fonctions pour lire et écrire le contenu des inodes, qui sont identiques à celles pour lire et écrire dans des fichiers, et qui sont utilisées pour lire le contenu des dossiers.

\section{Input : clavier et souris}

Nous gérons le clavier et la souris via l'interface PS/2

\section{Output : écran et BSOD}

Pour l'écran nous avons utilisé la specification VBE \cite{specVBE} et son
interface multiboot \cite{multiboot}.
Concrètement nous demandons, dans le header multiboot une résolution à GRUB, si
elle est disponible, il nous fournit un buffer vidéo linéaire mappé dans la
mémoire physique a cette résolution sinon, il semble expérimentalement se
rabattre sur du $800 \times 600$.

Le loader est chargé de vérifier que le mode fournit par le bootloader est celui
demandé par le noyau à savoir : Profondeur 32 bits, Bleu puis vert puis rouge
puis alpha.

Il nous suffit ainsi de d'écrire pixel par pixel dans ce buffer pour afficher à
l'écran. malheureusement cette méthode nécessite d'envoyer la totalité du
contenu de l'écran de la RAM à la VRAM à chaque rendu. Sur un ordinateur moyen,
cela nous limite à 60 fps si le noyau utilise 100\% du temps CPU pour le rendu
(et encore on a optimisé à coup d'instructions SSE).

\chapter{Interface logicielle}

\section{Organisation}

Nous gérons un système de processus similaire à Linux, il y a trois niveaux
\begin{itemize}
\item Groupe de processus : un groupe à un identifiant \verb$gid$, il contient
  un certain nombre de processus dont le leader du groupe qui le même \verb$pid$
  que le \verb$gid$ du groupe. (\verb$pgid$ sous Linux).
\item Processus : Une application, Un processus dispose d'un identifiant
  \verb$pid$, d'un espace mémoire dédié, séparé des autres processus, d'une
  liste de descripteurs de fichiers et d'un certain nombre de threads (au moins
  un) dont un principal qui a le même \verb$tid$ que le \verb$pid$ de son processus. 
\item Thread : Un \og fil \fg{} d'exécution, i.e un état du processeur, Un
  thread partage tout avec les autres threads de son processus sauf les valeurs
  de ses registres. L'ordre d'exécution de Thread est organisé par le Scheduler. 
\end{itemize}

De plus les processus forment une hiérarchie, un arbre dont la racine est le
processus \verb$init$ et les autres processus sont des noeud ou des feuilles.

\section{Appels systèmes}

La plupart de nos appel système son similaire à Linux, les différences sont précisée.

\paragraph{Lecture/écriture}

Nous supportons actuellement read write  (poll)

\paragraph{Manipulation de processus} Nous supportons fork, wait, exit (exit\_group sous Linux),
texit (exit sous Linux), exec (execve sous Linux). Nous supportons également une
version \textit{extrement} allégée de clone pour la création de Thread.

\paragraph{fichiers} nous supportons open, close et seek selon la spécification Linux.

\paragraph{grpahique} openwin opentwin

\paragraph{communication inter processus :} nous supportons actuellement pipe,
dup et dup2 


La specification complète des appels système est dans la documentation de \verb$Syscalls.h$.

\section{Interface de flux et descripteur de fichier}

\paragraph{Flux :} Nous utilisons une classe virtuelle pure \verb$Stream$ qui représente n'importe
quel flux. Il possède certaine capacité décrite par un masque et tout objet du
projet qui a besoin d'être un flux pour un programme utilisateur passe par cette
interface. La documentation de cette classe apportera plus de détail.

\paragraph{Descripteur de fichier :} Un descripteur de ficher peut dans Super
OS, être soit un flux, soit un dossier, soit un fenêtre (qui a également une
interface de flux). Les appels système testent ce qu'est chaque appel système
avant d'effectuer l'opération correspondante.




\section{Système de fichier virtuel (VFS)}

TODO théophile

\section{Gestionaire de fenêtre et d'événement}

\paragraph{Fenêtres :} Le gestionnaire de fenêtre est codé en dur dans le noyau pour des raison de
simplicité.
Il est composé de 3 Types d'éléments :
\begin{itemize}
\item L'écran : (classe \verb$Screen$) Qui gère le buffer VBE, le buffer
  interne pour le double buffering et les fonction de transfert rapide entre les
  2.
\item Les Workspaces : Les workspaces représentent un écran virtuel, on peut
  passer de l'un à l'autre avec des raccourci clavier (Ctrl + Fi pour le numéro
  i).
\item Les fenêtres : Les fenêtre appartiennent à un unique workspace, elle ont
  une position et une taille, il y en a de 2 types
  \begin{itemize}
    \item Les fenêtres graphiques : elles fournissent une API graphique bas
      niveau (événement atomique, dessin au pixel)
    \item Les fenêtre textes : En Bref, un émulateur de terminal en Kernel space.
  \end{itemize}
\end{itemize}

\paragraph{Événements :} Les événements sont émis par le clavier et la souris,
il sont envoyé au workspace actif puis à la fenêtre correspondante puis au
programme correspondant (On n'a pas eu le temps d'implémenter la dernière étape
encore, ça sera fait pour jeudi).

À chaque étape, les événements peuvent être interceptés, pour effectuer une
action, genre changer de workspace, passer une fenêtre dans un autre workspace, etc.

\section{Attente}

L'attente des Threads (wait, read bloquant, ...) est géré par le couple de
classes Waiting, Waitable. Lorsqu'un Thread doit attendre, On le met en mode
attente, lorsque l'événement attendu se produit, on passe en mode fin d'attente
et lorsque le Scheduler teste si le thread est entrain d'attendre, et si oui,
s'il doit arrêter d'attendre. 


\section{Bibliothèques standard}

\paragraph{libc :} On a réalisé rapidement les fonctions d'affichage (printf,
putchar, ...) de gestion de la mémoire (malloc, free).
Puis les wrappers des appels systèmes : open close

\paragraph{libc++ :} Nous avons développé les conteneurs de la bibliothèque
standard du C++ au fur et à mesure de nos besoins (Les conteneurs associatifs
ont été implémenté comme des vecteur triés).

\paragraph{Multi-threading :} Bien que le noyau autorise le multi-threading,
aucun système de contrôle (mutex, ...) n'a encore été mis en place. (peut-être pour jeudi).

\chapter{Software}

\section{Shell} On un shell fonctionel sous Linux dans \verb$user/ttsh$, on pas
eu le temps de porter tout les appels système et bibliothèque et de le porter,
ça sera fait pour jeudi.

\chapter{Difficultés rencontrées}

\section*{GCC et ses optimisations du futur}

Quand on était en 32-bit, lors d'un interruption, on mettait sur pile les registres à sauvegarder, puis on appellait une fonction qui prenait en argument une structure contenant les valeurs des différents registres.
Comme en C une telle structure est passée par copie, gcc a jugé qu'il pouvait la modifier pour gagner de la place sur la pile.
C'est gênant car après on dépile ces registres pour reprendre le cours de l'exécution après l'interruption.
On a essayé diverses combinaisons de \verb$volatile$ et \verb$const$ mais ça n'a rien changé.
Finalement, on a empilé \verb$rsp$ pour passer une référence en argument, comme ça gcc ne peut pas modifier nos registres
gcc qui utilise les variables passées en argument pour des variables temporaires alors que c'est les registres à remettre à la fin de l'interruption (même si c'est volatile const etc)

Dans le code du clavier, on a une queue qui contient les différents scancodes qui arrivent du clavier. Dans la fonction \verb$poll$ qui attends une touche, on avait le comportement suivant :
\begin{itemize}
    \item Tant que la queue est vide, attendre une interruption (\verb$asm volatile("hlt");$)
    \item Retirer le premier élément de la file (via un appel de fonction)
\end{itemize}

Pour retirer le premier élément de la file, on a un \verb$assert$ qui vérifie que la file n'est pas vide.
Comme gcc ne s'attends pas à ce qu'une interruption puisse changer la mémoire actuelle, il a optimisé le code en :
\begin{itemize}
    \item Si la queue est vide, faire une boucle infinie de \verb$hlt$
    \item Sinon, on retire le premier élément de la file, en inlinant la fonction en enlevant l'assert (qui est inutile puisque la file n'est pas vide)
\end{itemize}
Notre solution a été de compiler cette portion du code en \verb$-O0$ mais avec plus de recul, on aurait du utiliser \verb$asm volatile("hlt" : : : "memory")$

\section*{Stack overflows}

Au début, on avait notre pile juste au dessus du noyau. Un jour, on a eu un bug qui a fait un récursion infinie, et donc la pile a commencé à écrire sur les données et le code du noyau.

Ensuite, on a alloué proprement la pile dans la mémoire virtuelle, avec rien en dessous. Quand on faisait une récursion infinie, au bout d'un moment ça lançait une page fault. Donc ça poussait des informations sur la pile. Donc ça faisait un page-fault. Donc ça faisait une triple-fault, l'ordinateur redémarre.

\section*{Integer overflows}

On a eu plusieurs bugs dans le disque dur, où à des moments on utilisait des entiers 32-bit et pas 64-bits.
Plus précisément, on multipliait des entiers 32-bits et ça ne faisait pas des entiers 64 bits.
On a eu ce bug dans les cas suivants :

\begin{itemize}
    \item Bas niveau, quand on prenait un LBA qu'on multipliait par 512
    \item \verb$Partition::getSize$ qui renvoie un \verb$size_t$ mais qui l'obtient en faisant un entier 32-bit fois 512
    \item la variable \verb$blockSize$ de l'ext2 était en 32-bit, et les numéros de block sont en 32-bits
\end{itemize}

\section*{Petits oublis qui n'ont pas l'air important mais en fait si}

Il faut synchroniser le disque dur tous les 512 octets

Quand on modifie le paging, il faut faire \verb$invlpg$ pour invalider le cache sinon ça fait n'importe quoi

\section*{Le monde en dehors de l'émulateur}

Sur un vrai PC on avait des gros problème de lenteur car on avait activé le writethrough par défaut et donc ça écrivait tout le temps en RAM (sans passer par le cache)

Le driver PS/2 <-> usb fait n'importe quoi sur un certain PC

\section*{Le code 64-bit}

C'est difficile de mettre du code 32 bit dans un code 64 bit, et le linker 64-bit met tout le code dans le fichier après 2Mo alors que la convention multiboot a besoin d'un truc précis.
On a résolu ce problème en faisant un loader maison.

\section*{Autre}

Dans un vector, les objets sont pas forcément construits et on a eu plein de bugs à cause de ça.
Par la suite on a mis en place de tests unitaires.

Les spécifications pour la communication avec le disque dur étaient incohérentes

\verb$dynamic_cast$ ne marchait pas tout le temps, et on ne comprenait pas quand. On a construit un héritage diamant classique de test sur lequel il ne marchait pas, et en bougeant les structures à l'intérieur de kmain ça s'est mis à marcher...
La solution a été d'enlever tous les héritages en diamant qu'on avait pour pouvoir utiliser \verb$static_cast$

Lorsqu'on faisait l'écriture sur le disque dur, on testait avec \verb$fsck$ si on avait encore un système de fichiers valide. Pour la suppression, il fallait effacer toute trace du fichier sinon \verb$fsck$ faisait petit à petit toutes les étapes inverses de la suppression.


\bibliographystyle{plain}

\bibliography{report}






\end{document}
